
\begin{table}[H]
\centering
\begin{tabularx}{\textwidth}{|X|c|c|c|}
\hline
\multicolumn{4}{|c|}{\textbf{Sprint 0 – Inicio y definición}} \\\hline
Actividad & Días & Inicio & Fin \\\hline
Reunión inicial con Profesora Guía & 1 & 02‑04‑2025 & 02‑04‑2025 \\
Documento de alcance y backlog & 5 & 03‑04‑2025 & 07‑04‑2025 \\
Inventario preliminar de datos (Business/Data Understanding) & 3 & 08‑04‑2025 & 10‑04‑2025 \\
\textbf{Entrega \#1 – Propuesta de trabajo de título} & 1 & 11‑04‑2025 & 11‑04‑2025 \\\hline

\multicolumn{4}{|c|}{\textbf{Sprint 1 – Exploración y limpieza}} \\\hline
Reunión Profesora Guía & 1 & 16‑04‑2025 & 16‑04‑2025 \\
Análisis exploratorio de datos (EDA) & 9 & 17‑04‑2025 & 25‑04‑2025 \\
Limpieza y balanceo de datos (Data Preparation) & 7 & 30‑04‑2025 & 06‑05‑2025 \\
Reunión revisión con Profesora Guía & 1 & 07‑05‑2025 & 07‑05‑2025 \\\hline

\multicolumn{4}{|c|}{\textbf{Sprint 2 – Modelamiento inicial}} \\\hline
Ajustes finales al sprint & 1 & 08‑05‑2025 & 08‑05‑2025 \\
\textbf{Entrega \#2 – Análisis} & 1 & 09‑05‑2025 & 09‑05‑2025 \\
Benchmark de modelos base (Modeling v1) & 14 & 14‑05‑2025 & 27‑05‑2025 \\\hline

\multicolumn{4}{|c|}{\textbf{Sprint 3 – Iteración de modelos}} \\\hline
Continuación benchmark & 5 & 28‑05‑2025 & 01‑06‑2025 \\
Reunión revisión modelos con Profesora Guía & 1 & 04‑06‑2025 & 04‑06‑2025 \\
Preparación informe preliminar & 1 & 05‑06‑2025 & 05‑06‑2025 \\\hline

\multicolumn{4}{|c|}{\textbf{Sprint 4 – Integración y diseño}} \\\hline
\textbf{Entrega \#3 – Diseño} & 1 & 06‑06‑2025 & 06‑06‑2025 \\
Integración segmentación U‑Net & 10 & 11‑06‑2025 & 20‑06‑2025 \\
Optimización segmentación‑clasificación & 3 & 21‑06‑2025 & 23‑06‑2025 \\
Reunión revisión segmentación & 1 & 24‑06‑2025 & 24‑06‑2025 \\\hline

\multicolumn{4}{|c|}{\textbf{Sprint 5 – Evaluación}} \\\hline
Validación interna y métricas & 8 & 25‑06‑2025 & 02‑07‑2025 \\
Reunión revisión resultados & 1 & 02‑07‑2025 & 02‑07‑2025 \\
Ajustes finales + reporte & 6 & 03‑07‑2025 & 08‑07‑2025 \\\hline

\multicolumn{4}{|c|}{\textbf{Sprint 6 – Cierre}} \\\hline
\textbf{Entrega \#4 – Implementación} & 1 & 11‑07‑2025 & 11‑07‑2025 \\
Retrospectiva y planificación Seminario II & 1 & 22‑07‑2025 & 22‑07‑2025 \\\hline
\multicolumn{4}{|r|}{\textbf{Duración total: 83 días (02‑04‑2025 al 22‑07‑2025)}} \\\hline
\end{tabularx}
\caption{Cronograma iterativo por Sprints para el desarrollo del Seminario I bajo metodología ágil Scrum.}
\label{tbl:cronograma_scrum}
\end{table}


%\footnotetext{%
%\textbf{EDA} (\emph{Exploratory Data Analysis}): fase inicial dedicada a describir y visualizar el conjunto de datos para detectar patrones, anomalías y relaciones antes de aplicar modelos.%
%}
%\footnotetext{%
%\textbf{U‑Net}: Arquitectura de red neuronal convolucional con forma de “U” compuesta por un encoder (contracción) y un decoder (expansión) conectados mediante \emph{skip connections}. Diseñada para segmentación semántica de imágenes, permite generar una máscara por píxel aun con conjuntos de datos biomédicos relativamente pequeños.%
%}
