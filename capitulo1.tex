
\section{Introducción}
\label{sec:Intr}

La diabetes mellitus es reconocida mundialmente como uno de los principales desafíos sanitarios del siglo XXI, debido a su elevada prevalencia -- que se a duplicado entre 2003 y 2016-2017, pasando de un 6,4\% a un 12,3\%  de la poblacion en la Encuesta Nacional de Salud (ENS) \cite{troncoso2020estilos} -- y a las múltiples complicaciones asociadas que afectan significativamente la calidad de vida de los pacientes, así como los costos para los sistemas de salud \cite{ministerio2022estrategia}. Esta enfermedad crónica no solo genera un impacto directo sobre quienes la padecen, sino también sobre sus familias y comunidades, debido al manejo continuo y a largo plazo que exige. Entre dichas complicaciones, las ulceras en pacientes con pie diabetico destacan por su alta incidencia, gravedad clinica y complejidad terapeutica. Estas lesiones suponene una carga considerable para los sistemas de salud, dada su tendencia a la cronicidadm las frecuentes recaidas y el riesgo elevado de infecciones graves que, en muchos casos, culminan en amputaciones subsecuentes. De hecho, las heridas en pie diabetico son una de las principales causas de hospitalización prolongada y discapacidad permanente en estos pacientes \cite{mishra2017diabetic,bandyk2018diabetic}.

La evolucion clinica de las heridas ulcerosas en pie diabetico depende en gran medida de criterios visuales subjetivos, lo que genera inconsistencias diagnosticas y retrasos en  la toma de decisiones terapeuticas oportunas. La falta de criterios cuantitativos y estandarizados dificulta la comparacion entre distintos profesionales y centros, y retrasa la optimizacion temprana de los tratamientos. Este retraso agrava el pronostico y eleva el coste asistencial, pues las ulceras que no cicatrizan en las primeras cuatro semanas se cronifican con mas facilidad y requieren intervenciones mas invasivas y costosas \cite{mishra2017diabetic,bandyk2018diabetic}. Por ello, resulta imprescindible avanzar hacia metodos de evaluacion que integren herramientas objetivas -- capaces de monitorizar el tamaño, la profundidad y la composicion tisular de la lesion -- y que ofrezcan un apoyo real a la decision clinica. En este contexto, la inteligencia artificial aplicada al analisis de imagenes y la automatizacion de escalas de valorizacion como Photographic Wound Assessment Tool (PWAT) \cite{thompson2013reliability} representan una via prometedora para mejorar la precision, la reproducibilidad y la rapidez en el seguimiento de la cicatrizacion.

Paralelamente, esta problemática se encuentra agravada por el fenómeno de la multimorbilidad, definida como la coexistencia de dos o más condiciones crónicas en un mismo individuo, lo cual implica un incremento sustancial en la complejidad clínica y la carga asistencial, demandando enfoques integrales y basados en evidencia que optimicen los resultados en salud \cite{sarmiento2022patrones}.

En respuesta a este contexto, surge la necesidad de desarrollar herramientas objetivas y cuantitativas para mejorar el seguimiento y la evaluación del proceso de cicatrización en heridas ulcerosas del pie diabético. Esta investigación y desarrollo propone enfrentar dicha necesidad mediante el desarrollo de una herramienta automatizada que estime objetivamente el estadio de estas heridas a partir de técnicas avanzadas de procesamiento de imágenes y aprendizaje automático. La herramienta propuesta integrará métodos de segmentación automática de heridas utilizando imágenes capturadas mediante dispositivos móviles, la extracción y análisis de características radiómicas empleando técnicas especializadas como pyradiomics \cite{van2017computational}, y algoritmos avanzados de machine learning para una estimación precisa del estadio clínico según la escala PWAT (Photographic Wound Assessment Tool) \cite{thompson2013reliability}. Además, se contemplará la incorporación de información clínica relevante del paciente mediante registros digitales, buscando así una evaluación integral que permita una gestión clínica más eficiente.

De esta manera, la implementación de esta herramienta busca mejorar significativamente la precisión diagnóstica, reducir complicaciones mayores, favorecer intervenciones oportunas y eficientes, y optimizar el uso de recursos sanitarios disponibles, respondiendo así efectivamente a un desafío sanitario de creciente relevancia global \cite{organizacion2016informe}.