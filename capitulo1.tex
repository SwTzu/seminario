
\section{Introducción}
\label{sec:Intr}
La diabetes mellitus es reconocida mundialmente como uno de los principales desafíos sanitarios del siglo XXI, debido a su elevada prevalencia y a las múltiples complicaciones asociadas que afectan significativamente la calidad de vida de los pacientes, así como los costos para los sistemas de salud \cite{ministerio2022estrategia}. Esta enfermedad crónica no solo genera un impacto directo sobre quienes la padecen, sino también sobre sus familias y comunidades, debido al manejo continuo y a largo plazo que exige. Las complicaciones derivadas de la diabetes, tales como enfermedades cardiovasculares, nefropatías y neuropatías, incrementan significativamente la morbilidad y disminuyen la esperanza de vida de los pacientes.

Entre estas complicaciones, las heridas ulcerosas en pacientes con pie diabético destacan por su alta incidencia, gravedad clínica y complejidad en el manejo terapéutico. Dichas heridas representan una carga considerable para los sistemas de salud debido a su tendencia a la cronicidad, las frecuentes recaídas y el alto riesgo de infecciones graves que pueden culminar en amputaciones subsecuentes \cite{mishra2017diabetic,bandyk2018diabetic}. De hecho, las heridas en pie diabético constituyen una de las principales causas de hospitalización prolongada y discapacidad permanente en estos pacientes.

Actualmente, la evaluación clínica de las heridas ulcerosas depende predominantemente de criterios visuales subjetivos, lo cual genera inconsistencias en los diagnósticos, además de retrasos en la toma de decisiones terapéuticas oportunas. Esta problemática cobra aún mayor relevancia dada la tendencia creciente de la diabetes mellitus registrada por diversos estudios nacionales. Según datos proporcionados por la Encuesta Nacional de Salud (ENS), la prevalencia de diabetes aumentó significativamente desde 6,4\% en 2003 hasta alcanzar un 12,3\% durante el período 2016-2017, evaluado mediante glicemia en ayuno \cite{troncoso2020estilos}. Este incremento evidencia una alarmante tendencia al alza en la incidencia de diabetes, que se traduce en una mayor demanda asistencial.

Paralelamente, esta problemática se encuentra agravada por el fenómeno de la multimorbilidad, definida como la coexistencia de dos o más condiciones crónicas en un mismo individuo, lo cual implica un incremento sustancial en la complejidad clínica y la carga asistencial, demandando enfoques integrales y basados en evidencia que optimicen los resultados en salud \cite{sarmiento2022patrones}.

En respuesta a este contexto, surge la necesidad de desarrollar herramientas objetivas y cuantitativas para mejorar el seguimiento y la evaluación del proceso de cicatrización en heridas ulcerosas del pie diabético. Esta investigación y desarrollo propone enfrentar dicha necesidad mediante el desarrollo de una herramienta automatizada que estime objetivamente el estadio de estas heridas a partir de técnicas avanzadas de procesamiento de imágenes y aprendizaje automático. La herramienta propuesta integrará métodos de segmentación automática de heridas utilizando imágenes capturadas mediante dispositivos móviles, la extracción y análisis de características radiómicas empleando técnicas especializadas como pyradiomics \cite{van2017computational}, y algoritmos avanzados de machine learning para una estimación precisa del estadio clínico según la escala PWAT (Photographic Wound Assessment Tool) \cite{thompson2013reliability}. Además, se contemplará la incorporación de información clínica relevante del paciente mediante registros digitales, buscando así una evaluación integral que permita una gestión clínica más eficiente.

De esta manera, la implementación de esta herramienta busca mejorar significativamente la precisión diagnóstica, reducir complicaciones mayores, favorecer intervenciones oportunas y eficientes, y optimizar el uso de recursos sanitarios disponibles, respondiendo así efectivamente a un desafío sanitario de creciente relevancia global \cite{organizacion2016informe}.